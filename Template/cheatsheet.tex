\documentclass[a4paper, 8pt, landscape]{cheatJTS}

\usepackage[english]{babel}

\graphicspath{ {Images/} }

\begin{document}
    \makeatletter
        \defVal{title}{Linux Terminal}
        \defVal{author}{Jonathan Tobias da Silva}
        \defVal{date}{\today}
        \defVal{subject}{Assunto do cheat sheet}
        \defVal{keywords}{a, b, c, d}
        \defVal{logo}{\includegraphics[height=1.2cm]{logoExample.png}}
    \makeatother

    \begin{cheatsheet}

        \section{Section}
        \subsection{Subsection}
        \subsubsection{Subsubsection}

        \section{Section}
        \begin{topics}
            Some values & Some values \\
            Some values & Some values \\
            Some values & Some values \\
            Some values & Some values \\
            Some values & Some values \\
            Some values & Some values \\
            Some values & Some values \\
            Some values & Some values \\
            Some values & Some values \\
            Some values & Some values \\
            Some values & Some values \\
            Some values & Some values \\
            Some values & Some values \\
            Some values & Some values \\
            Some values & Some values \\
            Some values & Some values \\
            Some values & Some values \\
            Some values & Some values \\
            Some values & Some values \\
            Some values & Some values \\
            Some values & Some values \\
            Some values & Some values \\
            Some values & Some values \\
            Some values & Some values \\
            Some values & Some values \\
            Some values & Some values \\
            Some values & Some values \\
            Some values & Some values \\
            Some values & Some values \\
            Some values & Some values \\
            Some values & Some values \\
            Some values & Some values \\
            Some values & Some values \\
            Some values & Some values \\
            Some values & Some values \\
            Some values & Some values \\
            Some values & Some values \\
            Some values & Some values \\
            Some values & Some values \\
            Some values & Some values \\
            Some values & Some values \\
            Some values & Some values \\
            Some values & Some values \\
            Some values & Some values \\
            Some values & Some values \\
            Some values & Some values \\
            Some values & Some values \\
            Some values & Some values \\
            Some values & Some values \\
            Some values & Some values \\
            Some values & Some values \\
        \end{topics}

        \section{The menukeys}
        \begin{onetopic}
            You can visualize paths \directory{/home/moose/Desktop/manual.tex} 
            or menus \menu{View > Highlight Mode > Markup > LaTeX} (\menu[,]{Extras,Settings,{Units, rulers and origin}}) or key press 
            combinations: \keys{\ctrl + \shift + \Alt + \cmdmac + F} is for formatting in Eclipse.

            You can also visualize \keys{\tab}, \keys{\capslock}, \keys{\Space}, \keys{\SPACE},
            \keys{\arrowkeyup} and many more.

            See all keys supported (use \textbackslash\{key\}mac and \textbackslash\{key\}win to change the os macro locally):
            \shift, \capslock, \capslockmac, \tab, \tabmac, \esc, \escmac, \oldescmac, \ctrl, 
            \Alt, \Altmac, \AltGr, \cmdmac, \Space, \SPACE, \return, \returnmac, \enter, \entermac, \winmenu, \backspace, 
            \del, \delmac, \backdel, \backdelmac, \arrowkey{^}, \arrowkeyup, \arrowkey{v}, \arrowkeydown, \arrowkey{>}, 
            \arrowkeyright, \arrowkey{<}, \arrowkeyleft
        \end{onetopic}

        % \section{The menukeys}
        % You can visualize paths \directory{/home/moose/Desktop/manual.tex} 
        % or menus \menu{View > Highlight Mode > Markup > LaTeX} (\menu[,]{Extras,Settings,{Units, rulers and origin}}) or key press 
        % combinations: \keys{\ctrl + \shift + \Alt + \cmdmac + F} is for formatting in Eclipse.

        % You can also visualize \keys{\tab}, \keys{\capslock}, \keys{\Space}, \keys{\SPACE},
        % \keys{\arrowkeyup} and many more.

        % See all keys supported (use \textbackslash\{key\}mac and \textbackslash\{key\}win to change the os macro locally):
        % \shift, \capslock, \capslockmac, \tab, \tabmac, \esc, \escmac, \oldescmac, \ctrl, 
        % \Alt, \Altmac, \AltGr, \cmdmac, \Space, \SPACE, \return, \returnmac, \enter, \entermac, \winmenu, \backspace, 
        % \del, \delmac, \backdel, \backdelmac, \arrowkey{^}, \arrowkeyup, \arrowkey{v}, \arrowkeydown, \arrowkey{>}, 
        % \arrowkeyright, \arrowkey{<}, \arrowkeyleft

        \section{Lipsum}
        \subsection{subLipsum1}
            \lipsum[1-5]
        \subsubsection{subsubLipsum1}
            \lipsum[1-5]
        \subsubsection{subsubLipsum1}
            \lipsum[1-5]
        \subsection{subLipsum2}
            \lipsum[1-5]
            
    \end{cheatsheet}

\end{document}